\chapter{Условие лабораторной}

Целью данной работы является разработка программы с графическим интерфейсом для моделирования системы массового обслуживания (СМО) при помощи принципа $\Delta t$ и событийного принципа и определения максимальной длины очереди, при которой не будет потери сообщений. Рассматриваемая СМО состоит из генератора сообщений, очереди ожидающих обработки сообщений и обслуживающего аппарата (ОА). Генерация сообщений происходит по равномерному закону распределения, время обработки сообщений --- согласно нормальному распределению. Необходимо предоставить возможности ручного задания необходимых параметров, а также возможности возврата обработанного сообщения в очередь обработки с заданной вероятностью.


\chapter{Теоретическая часть}

\section{Управляющая программа имитационной модели}

Если программа-имитатор от источника информации обслуживающего аппарата, буферной памяти отображает работу отдельных устройств, то управляющая программа имитирует алгоритм взаимодействия всех устройств системы.

Управляющая программа реализуется по следующим принципам.

\subsection*{Пошаговый принцип}

Принцип $\Delta t$ заключается в последовательном анализе состояний всех блоков системы в момент времени $t + \Delta t$ по заданному состоянию блоков в момент времени t. При этом новое состояние определяется в соответствии с их алгоритмическим описанием с учетом случайных факторов, задаваемых распределениями вероятности. В результате этого решения проводится анализ, позволяющий определить, какие общесистемные события должны имитироваться в программной модели на данный момент времени.

Основной недостаток принципа $\Delta t$ --- значительные затраты машинного времени на анализ и контроль правильности функционирования всей системы. При недостаточно малом $\Delta t$ появляется опасность пропуска отдельных событий в системе, что исключает возможность получения правильных результатов примоделировании.

\subsection*{Событийный принцип}

Характерное свойство моделируемых систем обработки информации --- то, что состояния отдельных устройств изменяются в дискретные моменты времени, совпадающие с моментами поступления сообщений в систему, окончания решения той или иной задачи, возникновения аварийных сигналов. Поэтому моделирование и продвижение текущего времени в системе удобно производить, используя событийный принцип, при реализации которого состояния всех блоков имитационной (программной) модели анализируется лишь в момент появления какого-либо события. 
Момент наступления следующего события определяется минимальным значением из списка будущих событий, представляющего собой совокупность ближайшего изменения состояния каждого из блоков системы.

\section{Используемые законы распределения}

\subsection*{Закон появления сообщений}

Согласно заданию лабораторной работы для генерации сообщений используется равномерный закон распределения.
Случайная величина $X$ имеет \textit{равномерное распределение} на отрезке~$[a,~b]$, если ее плотность распределения~$f(x)$ равна:
\begin{equation}
	p(x) =
	\begin{cases}
		\displaystyle\frac{1}{b - a}, & \quad \text{если } a \leq x \leq b;\\
		0,  & \quad \text{иначе}.
	\end{cases}
\end{equation}

При этом функция распределения~$F(x)$ равна:

\begin{equation}
	F(x) =
	\begin{cases}
		0,  & \quad x < a;\\
		\displaystyle\frac{x - a}{b - a}, & \quad a \leq x \leq b;\\
		1,  & \quad x > b.
	\end{cases}
\end{equation}

Обозначение: $X \sim R[a, b]$.

Интервал времени между появлением $i$-ого и $(i - 1)$-ого сообщения по равномерному закону распределения вычисляется следующим образом:

\begin{equation}
	T_{i} = a + (b - a) \cdot R,
\end{equation}

\noindent где $R$ --- псевдослучайное число от 0 до 1.

\subsection*{Закон обработки сообщений}

Для моделирования работы генератора сообщений в лабораторной работе используется Нормальное распределение.
Cлучайная величина $X$ имеет \textit{нормальное распределение} с
параметрами~$m$~и~$\sigma$, если ее плотность распределения~$f(x)$ равна:

\begin{equation}
	f(x) = \frac{1}{\sigma \cdot \sqrt{2\pi}}~~e^{\displaystyle-\frac{(x -
			m)^2}{2\sigma^2}}, \quad x \in \mathbb{R}, \sigma > 0.
\end{equation}

При этом функция распределения~$F(x)$ равна:

\begin{equation}
	F(x) = \frac{1}{\sigma \cdot \sqrt{2\pi}} \int\limits_{-\infty}^{x}
	e^{\displaystyle-\frac{(t - m)^2}{2\sigma^2}} dt,
\end{equation}

или, что то же самое:

\begin{equation}
	F(x) = \frac{1}{2} \cdot \bigg[1 + erf\bigg(\frac{x - m}{\sigma
		\sqrt{2}}\bigg)\bigg],
\end{equation}

где $erf(x) = $ \scalebox{1.3}{$\frac{2}{\sqrt{\pi}}$}
\scalebox{1.1}{$\int\limits_{0}^{x} e^{ -t^2} dt$} --- функция вероятности
ошибок.

Обозначение: $X \sim N(m, \sigma^2)$. 

Интервал времени между появлением $i$-ого и $(i - 1)$-ого сообщения по нормальному распределению вычисляется следующим образом:

\begin{equation}
	T_i = \sigma \sqrt\frac{12}{n}(\sum\limits_{i=1}^{n} R_i - \frac{n}{2}) + m,
\end{equation}

\noindent где $R_i$ --- псевдослучайное число от 0 до 1.
